%Compile Recipe: XeLaTeX->biber->XeLaTeX->XeLaTeX

\documentclass[11pt]{SWJTUBeamer}

\usepackage[backend=biber,style=gb7714-2015,sorting=none]{biblatex}
\addbibresource{ref.bib}%add the bib file for bibtex,if you don't want to use bibliography,please comment this line and the line below
\beamertemplatetextbibitems

\title[SWJTU Beamer Template]{Beamer Template\\for SWJTUers}%words in the [] will be displayed on the bottom of every page,while the one in the {} will be displayed in the titlepage
\subtitle{西南交通大学\\Beamer 模板}
\author[User]{User}%the same to the "title"
\institute[School]{School}%the same to the "title"
\date{date} %Activate to display a given date or no date (if empty),otherwise the current date is printed

\begin{document}

%----------------------------------------------------------------------
% Title frame
\begin{frame}
    \maketitle
\end{frame}

\section{Generality}

\begin{frame}{Ordered List}
    \begin{enumerate}
        \item First
        \item Second
              \begin{enumerate}
                  \item First
                  \item Second
                  \item ...
              \end{enumerate}
        \item ...
    \end{enumerate}
\end{frame}

\begin{frame}{Unordered List}
    \begin{itemize}
        \item First
        \item Second
        \item ...
    \end{itemize}
\end{frame}

\begin{frame}
    \frametitle{Timeline}

    \begin{table}
        \centering
        \renewcommand\arraystretch{1.4}\arrayrulecolor{jtublue}
        \captionsetup{singlelinecheck=false, labelfont=sc, labelsep=quad}
        \caption{Timeline}\vskip -1.5ex
        \begin{tabular}{@{\,}r <{\hskip 2pt} !{\foo} >{\raggedright\arraybackslash}p{5cm}}
            \toprule
            \addlinespace[0.5ex]
            2020.6-2020.7   & Topic selection        \\
            2020.8-2020.9   & Literature Review      \\
            2020.10-2020.12 & Data collection        \\
            2021.1-2021.2   & Write academic papers  \\
            2021.3-2021.5   & The concluding defense \\
        \end{tabular}
    \end{table}

\end{frame}


\section{Methods}

\begin{frame}{Block}
    \begin{block}{Part 1}
        Test.
    \end{block}
    \begin{theorem}[Thm 1]
        Thm.
    \end{theorem}
    \begin{proof}
        Q.E.D.
    \end{proof}
\end{frame}

\begin{frame}{Enumerate}
    \begin{equation}
        growth=\beta_{0}+\beta_{1}hsr_{it}+\beta_{2}t_{it}+\beta_{3}hsr_{it}*t_{it}+\sum^{6}_{k=1}\alpha_{k}X_{it}+u_{i}+\varepsilon_{it}\label{eq1}
    \end{equation}
    \begin{enumerate}
        \item First \redstress{important}
        \item Second \eqref{eq1}
    \end{enumerate}
\end{frame}

\begin{frame}
    \frametitle{Tikzpicture}

    \begin{figure}[!t]
        \centering
        \resizebox{0.75\textheight}{!}{
            \begin{tikzpicture}
                [mindmap,
                    every node/.style={concept, execute at begin node=\hskip0pt, font=\Large\bfseries},
                    root concept/.append style={concept color=jtublue, fill=white, line width=1ex, text=jtublue, font=\huge\bfseries},
                    method a/.style={concept color=red, line width=1ex, text=white, font=\Large\bfseries},
                    method b/.style={concept color=orange, line width=1ex, text=white, font=\Large\bfseries},
                    methods a/.style={concept color=red!50, line width=1ex, text=white, font=\Large\bfseries},
                    methods b/.style={concept color=orange!50, line width=1ex, text=white, font=\Large\bfseries},
                    method c/.style={concept color=red, line width=1ex, text=white, font=\Large\bfseries},
                    method d/.style={concept color=orange, line width=1ex, text=white, font=\Large\bfseries},
                    methods c/.style={concept color=red!50, line width=1ex, text=white, font=\Large\bfseries},
                    methods d/.style={concept color=orange!50, line width=1ex, text=white, font=\Large\bfseries},
                    grow cyclic,
                    level 1/.append style={level distance=4.5cm, sibling angle=45},
                    level 2/.append style={level distance=3.5cm, sibling angle=75}]
                \node [root concept] {HSR\&\\Econ.} % root
                child [missing] {}
                child [method a] { node {Combined Effect}
                        child [methods a] { node {Returns to Scale} }
                        child [methods a] { node {Externality} }
                        child [methods a] { node {Spatial Competition} } }
                child [missing] {}
                child [method b] { node {Spillover Effect} }
                child [missing] {}
                child [method c] { node {Corridor Effect} }
                child [missing] {}
                child [method d] { node {Time-Lag Effect} };
            \end{tikzpicture}
        }
    \end{figure}

\end{frame}

\section{Conclusion}

\begin{frame}{Algorithm}
    \begin{algorithm2e}[H]
        \caption{Algorithm 1}\label{alg:em}
        \begin{algorithmic}[1]
            \REQUIRE Param
            \ENSURE $a$
            \REPEAT
            \STATE Compute $a_n$
            \UNTIL convergence
            \RETURN $a\leftarrow a_n$
        \end{algorithmic}
    \end{algorithm2e}
\end{frame}

\begin{frame}{Images}
    \begin{figure}
        \centering
        \includegraphics[width=0.85\textwidth]{swjtu_logo.png}
        \caption{SWJTU}\label{fig:jtu}
    \end{figure}
\end{frame}

\begin{frame}{Columns}
    \begin{columns}
        \begin{column}{0.45\textwidth}
            \begin{figure}
                \centering
                \includegraphics[width=0.95\textwidth]{swjtu_logo.png}
                \caption{SWJTU}\label{fig:SWJTU}
            \end{figure}
        \end{column}
        \begin{column}{0.55\textwidth}
            \begin{itemize}
                \item First
                \item Second
                \item ...
            \end{itemize}
        \end{column}
    \end{columns}
\end{frame}

\begin{frame}{Subfigure}
    \begin{figure}
        \centering
        \subfigure[]{\includegraphics[width=0.47\textwidth]{swjtu_logo.png}}
        \subfigure[]{\includegraphics[width=0.47\textwidth]{swjtu_logo.png}}
        \caption{Subfigure\footnote{See: \url{https://www.swjtu.edu.cn/}}}\label{fig:subfig}
    \end{figure}
\end{frame}

% To put the content of a frame in several pages, use allowframebreaks
\begin{frame}[allowframebreaks]
    \frametitle{Longframe}
    \begin{itemize}
        \item First\cite{paper3}
        \item Second
        \item ...
    \end{itemize}
\end{frame}

\section{Promotion}

\begin{frame}{More block}
    \begin{exampleblock}{Example}
        Eg1.
    \end{exampleblock}
    \begin{alertblock}{Attention}
        Test
    \end{alertblock}
\end{frame}

\begin{frame}{Table}
    \begin{table}[]
        \centering
        \caption{Data}
        \label{tab1}
        \begin{tabular}{@{}ccccc@{}}
            \toprule
                    & $q$ & $r$ & $a$ & $p$ \\ \midrule
            Reality & $1$ & $5$ & 2   & 3   \\
            Method1 & $4$ & $3$ & 1   & 1   \\
            Method2 & $4$ & $3$ & 2   & 2   \\
            Method3 & $5$ & $2$ & 3   & 3   \\
            Method4 & $4$ & $2$ & 2   & 2   \\ \bottomrule
        \end{tabular}
    \end{table}
\end{frame}

\begin{frame}[fragile]
    \frametitle{Code}
    \begin{lstlisting}[language=c]
#include"stdio.h"
#include"stdlib.h"
#define _CRT_SECURE_NO_WARNINGS
int main(int argc, char* argv[ ], char* envp[ ])
    {
        printf("Hello,SWJTUers!");
        return 0;
    }
\end{lstlisting}
\end{frame}

\begin{frame}{Conclusion}
    \begin{description}
        \item[I] First of all
        \item[II] Besides
        \item[III] Last but not least\cite{name7}
    \end{description}
\end{frame}

\begin{frame}{Thanking}
    \begin{block}{Thanking}
        Thanks.
    \end{block}
\end{frame}

\begin{frame}[allowframebreaks]
    \frametitle{A Example of Bibliography}
    %\nocite{*}%show all the bibliography without citation
    \printbibliography
\end{frame}

\end{document}
